\chapter{Appendice}
\addtocontents{toc}{\protect\setcounter{tocdepth}{0}}

Di seguito è riportata la lista dei documenti allegati, i quali
sono poi inseriti a seguire.
\vspace*{-25mm}
\listofappendices

\newappendix{Arduino R3 Schematic}\label{app:r3-schematic}
Sorgente: \url{https://www.arduino.cc/en/uploads/Main/Arduino_Uno_Rev3-schematic.pdf}\\
visitato il 07-09-2022

\begin{center}
    \includegraphics%
        [height=.89\textwidth, angle=90]%
        {attachments/Arduino_Uno_Rev3-schematic.pdf}
\end{center}

\newappendix{Algoritmo di programmazione memoria Flash tramite DebugWire}\label{app:dw-flash-prog}

\begin{lstlisting}[language=AVR]
    ;L'algoritmo si aspetta i seguenti registri impostati dal debugger:
    ; r26 = const(3)
    ; r27 = const(1)
    ; r28 = const(5)
    ; r29 = const(0x40)
    ; r30 = const(page_address_l)
    ; r31 = const(page_address_h)


    ;clear page
    out SPMCSR, r26 ; PGERS | SPMEN
    spm

    ;write tmp buffer
    ldi r28, 0x11 ; RWWSRE | SPMEN => read while write read enable if bootloader support, else clear tmp buffer
    out SPMCSR, r28
    spm

wrt_buf:
    in r0, DWDR; read data from dw, sent by debugger
    in r1, DWDR; r0:r1 contains the word to be written
    out SPMCSR, r27 ; SPMEN
    spm
    adiw Z, 2
    ;debugger loops to wrt_buf until the full page is written.

    ;issue write
    out SPMCSR, r28 ; PGWRT | SPMEN
    spm
\end{lstlisting}

\newappendix{Algoritmo di programmazione memoria\\EEPROM tramite DebugWire}\label{app:dw-eeprom-prog}

\begin{lstlisting}[language=AVR]
    ;L'algoritmo si aspetta i seguenti registri impostati dal debugger:
    ; r28 = const(4)
    ; r29 = const(2)
    ; r30 = const(address_l)
    ; r31 = const(address_h)

    out EEARH, r31
    out EEARL, r30 ; set write address
    in r0, DWDR ; sent by debugger by dw

    out EEDR, r0 ; set data to be written
    out EECR, r28 ; EEMPE
    out EECR, r29 ; EEPE
\end{lstlisting}

\addtocontents{toc}{\protect\setcounter{tocdepth}{3}}